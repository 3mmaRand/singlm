% Options for packages loaded elsewhere
\PassOptionsToPackage{unicode}{hyperref}
\PassOptionsToPackage{hyphens}{url}
%
\documentclass[
]{book}
\usepackage{lmodern}
\usepackage{amssymb,amsmath}
\usepackage{ifxetex,ifluatex}
\ifnum 0\ifxetex 1\fi\ifluatex 1\fi=0 % if pdftex
  \usepackage[T1]{fontenc}
  \usepackage[utf8]{inputenc}
  \usepackage{textcomp} % provide euro and other symbols
\else % if luatex or xetex
  \usepackage{unicode-math}
  \defaultfontfeatures{Scale=MatchLowercase}
  \defaultfontfeatures[\rmfamily]{Ligatures=TeX,Scale=1}
\fi
% Use upquote if available, for straight quotes in verbatim environments
\IfFileExists{upquote.sty}{\usepackage{upquote}}{}
\IfFileExists{microtype.sty}{% use microtype if available
  \usepackage[]{microtype}
  \UseMicrotypeSet[protrusion]{basicmath} % disable protrusion for tt fonts
}{}
\makeatletter
\@ifundefined{KOMAClassName}{% if non-KOMA class
  \IfFileExists{parskip.sty}{%
    \usepackage{parskip}
  }{% else
    \setlength{\parindent}{0pt}
    \setlength{\parskip}{6pt plus 2pt minus 1pt}}
}{% if KOMA class
  \KOMAoptions{parskip=half}}
\makeatother
\usepackage{xcolor}
\IfFileExists{xurl.sty}{\usepackage{xurl}}{} % add URL line breaks if available
\IfFileExists{bookmark.sty}{\usepackage{bookmark}}{\usepackage{hyperref}}
\hypersetup{
  pdftitle={singlm: A simple introduction to GLM for analysing Poisson and Binomal responses in R},
  pdfauthor={Emma Rand},
  hidelinks,
  pdfcreator={LaTeX via pandoc}}
\urlstyle{same} % disable monospaced font for URLs
\usepackage{color}
\usepackage{fancyvrb}
\newcommand{\VerbBar}{|}
\newcommand{\VERB}{\Verb[commandchars=\\\{\}]}
\DefineVerbatimEnvironment{Highlighting}{Verbatim}{commandchars=\\\{\}}
% Add ',fontsize=\small' for more characters per line
\usepackage{framed}
\definecolor{shadecolor}{RGB}{248,248,248}
\newenvironment{Shaded}{\begin{snugshade}}{\end{snugshade}}
\newcommand{\AlertTok}[1]{\textcolor[rgb]{0.94,0.16,0.16}{#1}}
\newcommand{\AnnotationTok}[1]{\textcolor[rgb]{0.56,0.35,0.01}{\textbf{\textit{#1}}}}
\newcommand{\AttributeTok}[1]{\textcolor[rgb]{0.77,0.63,0.00}{#1}}
\newcommand{\BaseNTok}[1]{\textcolor[rgb]{0.00,0.00,0.81}{#1}}
\newcommand{\BuiltInTok}[1]{#1}
\newcommand{\CharTok}[1]{\textcolor[rgb]{0.31,0.60,0.02}{#1}}
\newcommand{\CommentTok}[1]{\textcolor[rgb]{0.56,0.35,0.01}{\textit{#1}}}
\newcommand{\CommentVarTok}[1]{\textcolor[rgb]{0.56,0.35,0.01}{\textbf{\textit{#1}}}}
\newcommand{\ConstantTok}[1]{\textcolor[rgb]{0.00,0.00,0.00}{#1}}
\newcommand{\ControlFlowTok}[1]{\textcolor[rgb]{0.13,0.29,0.53}{\textbf{#1}}}
\newcommand{\DataTypeTok}[1]{\textcolor[rgb]{0.13,0.29,0.53}{#1}}
\newcommand{\DecValTok}[1]{\textcolor[rgb]{0.00,0.00,0.81}{#1}}
\newcommand{\DocumentationTok}[1]{\textcolor[rgb]{0.56,0.35,0.01}{\textbf{\textit{#1}}}}
\newcommand{\ErrorTok}[1]{\textcolor[rgb]{0.64,0.00,0.00}{\textbf{#1}}}
\newcommand{\ExtensionTok}[1]{#1}
\newcommand{\FloatTok}[1]{\textcolor[rgb]{0.00,0.00,0.81}{#1}}
\newcommand{\FunctionTok}[1]{\textcolor[rgb]{0.00,0.00,0.00}{#1}}
\newcommand{\ImportTok}[1]{#1}
\newcommand{\InformationTok}[1]{\textcolor[rgb]{0.56,0.35,0.01}{\textbf{\textit{#1}}}}
\newcommand{\KeywordTok}[1]{\textcolor[rgb]{0.13,0.29,0.53}{\textbf{#1}}}
\newcommand{\NormalTok}[1]{#1}
\newcommand{\OperatorTok}[1]{\textcolor[rgb]{0.81,0.36,0.00}{\textbf{#1}}}
\newcommand{\OtherTok}[1]{\textcolor[rgb]{0.56,0.35,0.01}{#1}}
\newcommand{\PreprocessorTok}[1]{\textcolor[rgb]{0.56,0.35,0.01}{\textit{#1}}}
\newcommand{\RegionMarkerTok}[1]{#1}
\newcommand{\SpecialCharTok}[1]{\textcolor[rgb]{0.00,0.00,0.00}{#1}}
\newcommand{\SpecialStringTok}[1]{\textcolor[rgb]{0.31,0.60,0.02}{#1}}
\newcommand{\StringTok}[1]{\textcolor[rgb]{0.31,0.60,0.02}{#1}}
\newcommand{\VariableTok}[1]{\textcolor[rgb]{0.00,0.00,0.00}{#1}}
\newcommand{\VerbatimStringTok}[1]{\textcolor[rgb]{0.31,0.60,0.02}{#1}}
\newcommand{\WarningTok}[1]{\textcolor[rgb]{0.56,0.35,0.01}{\textbf{\textit{#1}}}}
\usepackage{longtable,booktabs}
% Correct order of tables after \paragraph or \subparagraph
\usepackage{etoolbox}
\makeatletter
\patchcmd\longtable{\par}{\if@noskipsec\mbox{}\fi\par}{}{}
\makeatother
% Allow footnotes in longtable head/foot
\IfFileExists{footnotehyper.sty}{\usepackage{footnotehyper}}{\usepackage{footnote}}
\makesavenoteenv{longtable}
\usepackage{graphicx,grffile}
\makeatletter
\def\maxwidth{\ifdim\Gin@nat@width>\linewidth\linewidth\else\Gin@nat@width\fi}
\def\maxheight{\ifdim\Gin@nat@height>\textheight\textheight\else\Gin@nat@height\fi}
\makeatother
% Scale images if necessary, so that they will not overflow the page
% margins by default, and it is still possible to overwrite the defaults
% using explicit options in \includegraphics[width, height, ...]{}
\setkeys{Gin}{width=\maxwidth,height=\maxheight,keepaspectratio}
% Set default figure placement to htbp
\makeatletter
\def\fps@figure{htbp}
\makeatother
\setlength{\emergencystretch}{3em} % prevent overfull lines
\providecommand{\tightlist}{%
  \setlength{\itemsep}{0pt}\setlength{\parskip}{0pt}}
\setcounter{secnumdepth}{5}
\usepackage{booktabs}
\usepackage[]{natbib}
\bibliographystyle{apalike}

\title{singlm: A simple introduction to GLM for analysing Poisson and Binomal responses in R}
\author{Emma Rand}
\date{2020-07-15}

\begin{document}
\maketitle

{
\setcounter{tocdepth}{1}
\tableofcontents
}
\hypertarget{intro}{%
\chapter*{Introduction}\label{intro}}
\addcontentsline{toc}{chapter}{Introduction}

expected audience, prerequisite knowledge but will include brief intros. first course in data analysis in which taught regression, t-tests and ANOVA as separate tests.

assumes you have used t.test and aov

\begin{Shaded}
\begin{Highlighting}[]
\CommentTok{# make a plot}
\CommentTok{# a string}
\NormalTok{word <-}\StringTok{ "hello"}
\NormalTok{x <-}\StringTok{ }\KeywordTok{rnorm}\NormalTok{(}\DecValTok{20}\NormalTok{)}
\NormalTok{y <-}\StringTok{ }\KeywordTok{rnorm}\NormalTok{(}\DecValTok{20}\NormalTok{)}
\KeywordTok{plot}\NormalTok{(x,y)}
\end{Highlighting}
\end{Shaded}

previous teaching
- response variables
- predictor variables
- choice of test depends on nature of these
- regression one contin; t tests one categorical with two levels; one-way anova one categorical with 2 or more levels; two-way anova two categorical each with two or more
- functions that you probably used in R

would have been taught that these tests (or models) need response variable that are continuous, in particular normally distributed.

the subject of this book is to teach you how to deal with variables that are Poisson or Binomially distributed.

what is a poisson response

what is a binomial response is

These are analysed with glm()

my experience is that people are often confused by the output in comparison to t.test().

what helps is recognise that that t tests and anova and regression all of these have the same underlying maths. they are linear models and can be analysed with lm()

and the output of lm is very similar to that of glm

Overview of the chapter contents
Chapter 1
Revisit regression, t-tests and one- and two-way ANOVA in each case briefly saying when you use them, how you've probably done them before then how you do them with lm(). describe the links between the outs so you can related the new information to your previous knowledge

code conventions used in the book

approach taken in the book:
we'll be using \texttt{tidyverse} \citep{tidyverse2019} packages.

scope of the book, what isn't covered

\hypertarget{revisit}{%
\chapter{What are linear models}\label{revisit}}

\hypertarget{introduction}{%
\section{Introduction}\label{introduction}}

\begin{itemize}
\tightlist
\item
  what is meant by a linear model
\item
  revise regression
\item
  revise t.tests
\item
  doing t.tests as linear models
\item
  linking output of t.test to lm
\item
  revise one way ANOVA
\item
  linking output of aov to lm
\item
  revise two way ANOVA
\item
  doing two way ANOVA as linear models
\item
  linking output of aov to lm
\item
  extensible - ancova design
\end{itemize}

\hypertarget{what-is-a-linear-model}{%
\section{What is a linear model?}\label{what-is-a-linear-model}}

\hypertarget{single-linear-regression}{%
\section{Single linear regression}\label{single-linear-regression}}

\hypertarget{t-tests}{%
\section{t-tests}\label{t-tests}}

\hypertarget{t.test}{%
\subsection{t.test()}\label{t.test}}

\hypertarget{lm}{%
\subsection{lm()}\label{lm}}

\hypertarget{link-between-the-outputs}{%
\subsection{link between the outputs}\label{link-between-the-outputs}}

\hypertarget{reporting-from-lm}{%
\subsection{reporting from lm()}\label{reporting-from-lm}}

including a figure

\hypertarget{one-way-anova}{%
\section{One-way ANOVA}\label{one-way-anova}}

\hypertarget{aov}{%
\subsection{aov()}\label{aov}}

\hypertarget{lm-1}{%
\subsection{lm()}\label{lm-1}}

\hypertarget{link-between-the-outputs-1}{%
\subsection{link between the outputs}\label{link-between-the-outputs-1}}

\hypertarget{post-hoc-for-lm}{%
\subsection{post-hoc for lm()}\label{post-hoc-for-lm}}

\hypertarget{reporting-from-lm-1}{%
\subsection{reporting from lm()}\label{reporting-from-lm-1}}

including a figure

\hypertarget{two-way-anova}{%
\section{Two-way ANOVA}\label{two-way-anova}}

\hypertarget{aov-1}{%
\subsection{aov()}\label{aov-1}}

\hypertarget{lm-2}{%
\subsection{lm()}\label{lm-2}}

\hypertarget{link-between-the-outputs-2}{%
\subsection{link between the outputs}\label{link-between-the-outputs-2}}

\hypertarget{post-hoc-for-lm-1}{%
\subsection{post-hoc for lm()}\label{post-hoc-for-lm-1}}

\hypertarget{reporting-from-lm-2}{%
\subsection{reporting from lm()}\label{reporting-from-lm-2}}

including a figure

\hypertarget{section}{%
\section{}\label{section}}

\hypertarget{pois}{%
\chapter{GLM for poisson responses}\label{pois}}

\hypertarget{pois-intro}{%
\section{intro}\label{pois-intro}}

some stuff introducing pois

\hypertarget{pois-build}{%
\section{build}\label{pois-build}}

some stuff about build pois

\hypertarget{pois-output}{%
\section{output}\label{pois-output}}

some stuff about pois output

\hypertarget{bino}{%
\chapter{GLM for binomial responses}\label{bino}}

\hypertarget{bino-intro}{%
\section{intro}\label{bino-intro}}

some stuff introducing bino

\hypertarget{bino-build}{%
\section{build}\label{bino-build}}

some stuff about build bino

\hypertarget{bino-output}{%
\section{output}\label{bino-output}}

some stuff about bino output

\hypertarget{summary}{%
\chapter{Summary}\label{summary}}

key points

where to go next

  \bibliography{refs/book.bib,refs/packages.bib}

\end{document}
